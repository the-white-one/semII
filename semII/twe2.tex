\documentclass[12pt,titlepage]{scrartcl}
\usepackage[ngerman]{babel}
\usepackage[utf8]{inputenc}
\usepackage[a4paper,lmargin={2.5cm},rmargin={2.5cm},tmargin={2.5cm},bmargin={2.5cm}]{geometry}
\usepackage{graphicx}
\usepackage[T1]{fontenc}
\usepackage[headsepline,footsepline, singlespacing=true]{scrlayer-scrpage}
\usepackage{mwe}
\usepackage{acronym}
\usepackage{blindtext}
\usepackage[hidelinks]{hyperref}
\usepackage{csquotes}
\usepackage{setspace}

\usepackage[
  backend=biber,
  style=ext-authoryear,
  maxcitenames=2, maxbibnames=999,
  giveninits=true,
  uniquename=init, uniquelist=false,
  articlein=false, innamebeforetitle=true,
  punctfont=true, dashed=false,
]{biblatex}

% !TEX root = twe2.tex

%Zitierstil latex beibringen
\DefineBibliographyStrings{ngerman}{
	andothers = {{et\,al\adddot}}
}

\newcommand{\fontForTheSeminar}{ppl}
\renewcommand*{\mkbibnamefamily}{\expandafter\MakeUppercase\expandafter}
\AtBeginBibliography{%
  \renewcommand*{\mkbibnamefamily}[1]{#1}}

\DeclareNameFormat{labelname}{%
  \ifnum\value{uniquename}=0\relax
    \usebibmacro{name:family}
      {\namepartfamily}
      {\namepartgiven}
      {\namepartprefix}
      {\namepartsuffix}%
  \else
    \usebibmacro{name:family-given}
      {\namepartfamily}
      {\namepartgiveni}
      {\namepartprefix}
      {\namepartsuffixi}%
  \fi
  \usebibmacro{name:andothers}}

\DeclareNameAlias{default}{family-given}
\DeclareNameAlias{sortname}{default}

\DeclareDelimAlias*[bib]{finalnamedelim}{multinamedelim}
\setlength{\bibhang}{0pt}

\DeclareNameWrapperFormat{sortname}{\mkbibbold{#1}}
\DeclareFieldFormat{biblabeldate}{\mkbibbold{\mkbibparens{#1}}}
\DeclareFieldFormat{journaltitle}{#1\isdot}
\DeclareFieldFormat*{title}{#1}

\DeclareDelimFormat[bib]{nametitledelim}{\addcolon\space}

\renewcommand*{\volnumdelim}{\addcomma\space}

\DeclareFieldFormat{pages}{#1}

\renewcommand*{\bibpagespunct}{\addcolon\ifentrytype{article}{}{\space}}

\DeclareDelimFormat{postnotedelim}{\addcolon}
\DeclareFieldFormat{postnote}{\mknormrange{#1}}
\setlength\bibitemsep{6pt}

\renewcommand*{\bibfont}{\normalsize}

%Schriftwahl
\newcommand{\changefont}[3]{\fontfamily{#1} \fontseries{#2} \fontshape{#3} \selectfont}


\addbibresource{bibtex/bib.bib}

\setlength{\parindent}{0pt}
\RedeclareSectionCommand[%
	beforeskip=32pt,
	afterskip=6pt,
	runin=false]{section}

\RedeclareSectionCommand[%
	beforeskip=12pt,
	afterskip=6pt,
	runin=false]{subsection}

\RedeclareSectionCommand[%
	beforeskip=12pt,
	afterskip=6pt,
	runin=false]{subsubsection}

\pagestyle{scrheadings}
\changefont{\fontForTheSeminar}{m}{n}
\linespread{1.3}
\flushbottom


\begin{document}
\changefont{\fontForTheSeminar}{m}{n}

% !TEX root = twe2.tex

\begin{titlepage}
\centering

\begingroup
    \Large{Fachhochschule Kiel}
    \par
\endgroup

\begingroup
    \large{Hochschule für Angewandte Wissenschaften \\
        Fachbereich Agrarwirtschaft \\
        Osterrönfeld}
    \par
\endgroup


\vspace{3cm}


\begingroup
    \Large
    \bfseries{Seminar II}
    \par
\endgroup

\vspace{0.1cm}

im Studienfach Landwirtschaft

\vspace{2cm}

\hrulefill

\vspace{0.5cm}
\begingroup
    \LARGE
    \bfseries{Leguminosen Nachsaat: zusätzliches Protein aus dem Grünland}
    \par
\endgroup
\vspace{0.5cm}
\begingroup
    \small
    Weggler, K., Thumm, U., Elsäßer, M. (2019)
    \par
\endgroup
\vspace{0.2cm}
\hrulefill
\vspace{2cm}

vorgelegt von:

\begingroup
    \large{Tibor Weiß}
    \par
\endgroup

\vspace{2cm}

betreut von:

\begingroup
    \large{Prof. Dr. Rainer Wulfes}
    \par
\endgroup

\vspace{1cm}

Osterrönfeld, im Oktober 2020

\end{titlepage}


%einige letzte Befehle für das Layout
\setlength{\parskip}{9pt} %Abstand nach Absatz
\newpage %neue Seite nach dem DEckblatt
\clearpairofpagestyles %alle Layouteinstellungen vom Deckblatt löschen
\changefont{\fontForTheSeminar}{m}{n} %Schrift aktualisieren
\ohead{\headmark} %Kopfzeile
\ofoot{\pagemark} %Fußzeile
\automark{section} %aktuelle Section wird in der Kopfzeile angezeigt
\renewcommand*{\sectionmarkformat}{} %Darstellung Section in der Kopfzeile angepasst
\setcounter{page}{2} %Seitenzähler auf 2 Stellen


%Inhaltsverzeichnis
\tableofcontents


%Tabellen- und Abbildungsverzeichnis
%\listoftables
%\listoffigures
%Anhangsverzeichnis TODO

%Abkürzungsverzeichnis - Abkürzungen eintragen
\section*{Abkürzungsverzeichnis}
\addcontentsline{toc}{section}{Abkürzungsverzeichnis}
\begin{acronym}[SCCharts]
\acro{CAU}{Christian-Albrechts-Universität zu Kiel}
\acro{DSL}{Domain Specific Language}
\acro{EMF}{Eclipse Modelling Frammework}
\acro{IUR}{Initialize-Update-Read Protocol}
\acro{KICO}{KIELER Compiler}
\acro{KIELER}{Kiel Integrated Enviroment for Layout Eclipse RichClient}
\acro{PDG}{Programm Dependency Graph}
\acro{PRETSY}{Precision-Timed Synchronus Reactive Processing}
\acro{SCPDG}{Sequential Constructive Programm Dependency Graph}
\acro{SCCharts}{Sequentially Constructive Charts}
\acro{SCG}{Sequentially Constructive Graph}
\acro{WCET}{Worst-Case Execution Time}
\acro{WCRT}{Worst-Case Reaction Time}
\end{acronym}
\newpage

\section{Einleitung}
\blindtext

\section{Literatur}
\blindtext

\subsection{John B. Goodenough}
\blindtext

\subsubsection{Nobelpreis}
\Blindtext

\subsubsection{Li-Ion Batterie}
\Blindtext

\subsection{Alfred Nobel}
\blindtext

\section{Genial}
Das \ac{EMF} wird genutzt um Modelle einfach und automatisiert zu erstellen \parencite[348\psq]{karg1969einfluss}.
Nach \textcite[347]{helmert2003n} spielt \ac{EMF} hierbei eine wichtige Rolle. \ac{KIELER} wird an der \ac{CAU} entwickelt \parencite[34\psqq]{weggler2050leguminosen} wie \textcite[200]{karg1965larvalsystematische} beschrieben hat.
\ac{IUR} ist ein wichtiger bestandteil von \ac{SCCharts} \parencites[235]{karg1985zwei}[88]{bellebaum2008rohricht}.
\subsection{Wissenwertes}
\blindtext

\subsection{Details}
\blindtext

%Literaturverzeichnis
\begin{singlespace} %einfacher Zeilenabstand im Literaturverzeichnis
\begin{flushleft} %linksbündig
\addcontentsline{toc}{section}{Literaturverzeichnis}
\printbibliography[title=Literaturverzeichnis]
\end{flushleft}
\end{singlespace}

%Anhang
\end{document}
