\documentclass[12pt,titlepage]{scrartcl}
\usepackage[ngerman]{babel}
\usepackage[utf8]{inputenc}
\usepackage[a4paper,lmargin={2.5cm},rmargin={2.5cm},tmargin={2.5cm},bmargin={2.5cm}]{geometry}
\usepackage{graphicx}
\usepackage[T1]{fontenc}
\usepackage[headsepline,footsepline, singlespacing=true]{scrlayer-scrpage}
\usepackage{mwe}
\usepackage{acronym}
\usepackage{blindtext}
\usepackage[hidelinks]{hyperref}
\usepackage{csquotes}
\usepackage{setspace}

\usepackage[
  backend=biber,
  style=ext-authoryear,
  maxcitenames=2, maxbibnames=999,
  giveninits=true,
  uniquename=init, uniquelist=false,
  articlein=false, innamebeforetitle=true,
  punctfont=true, dashed=false,
]{biblatex}

% !TEX root = twe2.tex

%Zitierstil latex beibringen
\DefineBibliographyStrings{ngerman}{
	andothers = {{et\,al\adddot}}
}

\newcommand{\fontForTheSeminar}{ppl}
\renewcommand*{\mkbibnamefamily}{\expandafter\MakeUppercase\expandafter}
\AtBeginBibliography{%
  \renewcommand*{\mkbibnamefamily}[1]{#1}}

\DeclareNameFormat{labelname}{%
  \ifnum\value{uniquename}=0\relax
    \usebibmacro{name:family}
      {\namepartfamily}
      {\namepartgiven}
      {\namepartprefix}
      {\namepartsuffix}%
  \else
    \usebibmacro{name:family-given}
      {\namepartfamily}
      {\namepartgiveni}
      {\namepartprefix}
      {\namepartsuffixi}%
  \fi
  \usebibmacro{name:andothers}}

\DeclareNameAlias{default}{family-given}
\DeclareNameAlias{sortname}{default}

\DeclareDelimAlias*[bib]{finalnamedelim}{multinamedelim}
\setlength{\bibhang}{0pt}

\DeclareNameWrapperFormat{sortname}{\mkbibbold{#1}}
\DeclareFieldFormat{biblabeldate}{\mkbibbold{\mkbibparens{#1}}}
\DeclareFieldFormat{journaltitle}{#1\isdot}
\DeclareFieldFormat*{title}{#1}

\DeclareDelimFormat[bib]{nametitledelim}{\addcolon\space}

\renewcommand*{\volnumdelim}{\addcomma\space}

\DeclareFieldFormat{pages}{#1}

\renewcommand*{\bibpagespunct}{\addcolon\ifentrytype{article}{}{\space}}

\DeclareDelimFormat{postnotedelim}{\addcolon}
\DeclareFieldFormat{postnote}{\mknormrange{#1}}
\setlength\bibitemsep{6pt}

\renewcommand*{\bibfont}{\normalsize}

%Schriftwahl
\newcommand{\changefont}[3]{\fontfamily{#1} \fontseries{#2} \fontshape{#3} \selectfont}


\addbibresource{bibtex/bib.bib}

\setlength{\parindent}{0pt}
\RedeclareSectionCommand[%
	beforeskip=32pt,
	afterskip=6pt,
	runin=false]{section}

\RedeclareSectionCommand[%
	beforeskip=12pt,
	afterskip=6pt,
	runin=false]{subsection}

\RedeclareSectionCommand[%
	beforeskip=12pt,
	afterskip=6pt,
	runin=false]{subsubsection}

\pagestyle{scrheadings}
\changefont{\fontForTheSeminar}{m}{n}
\linespread{1.3}
\flushbottom


\begin{document}
\changefont{\fontForTheSeminar}{m}{n}

% !TEX root = twe2.tex

\begin{titlepage}
\changefont{\fontForTheSeminar}{m}{n}
\centering

\begingroup
    \Large{Fachhochschule Kiel}
    \par
\endgroup

\begingroup
    \large{Hochschule für Angewandte Wissenschaften \\
        Fachbereich Agrarwirtschaft \\
        Osterrönfeld}
    \par
\endgroup


\vspace{3cm}


\begingroup
    \Large
    \bfseries{Seminar II}
    \par
\endgroup

\vspace{0.1cm}

im Studienfach Landwirtschaft

\vspace{2cm}

\hrulefill

\vspace{0.5cm}
\begingroup
    \LARGE
    \bfseries{Leguminosen für zusätzliches Protein aus dem Grünland}
    \par
\endgroup
\vspace{0.5cm}
\hrulefill
\vspace{2cm}

vorgelegt von:

\begingroup
    \large{Tibor Weiß}
    \par
\endgroup

\vspace{1.5cm}

betreut von:

\begingroup
    \large{Prof. Dr. Rainer Wulfes \\
        Prof. Dr. John B. Goodenough}
    \par
\endgroup

\vspace{2cm}

Osterrönfeld, im Oktober 2020

\end{titlepage}


%einige letzte Befehle für das Layout
\setlength{\parskip}{9pt} %Abstand nach Absatz
\newpage %neue Seite nach dem DEckblatt
\clearpairofpagestyles %alle Layouteinstellungen vom Deckblatt löschen
\changefont{\fontForTheSeminar}{m}{n} %Schrift aktualisieren
\ohead{\headmark} %Kopfzeile
\ofoot{\pagemark} %Fußzeile
\automark{section} %aktuelle Section wird in der Kopfzeile angezeigt
\renewcommand*{\sectionmarkformat}{} %Darstellung Section in der Kopfzeile angepasst
\setcounter{page}{2} %Seitenzähler auf 2 Stellen


%Inhaltsverzeichnis
\tableofcontents


%Tabellen- und Abbildungsverzeichnis
%\listoftables
%\listoffigures
%Anhangsverzeichnis TODO

%Abkürzungsverzeichnis - Abkürzungen eintragen
\section*{Abkürzungsverzeichnis}
\addcontentsline{toc}{section}{Abkürzungsverzeichnis}
\begin{acronym}[AGGF] %die längste Abkürzung in die eckigen Klammern schreiben für tabellarische Darstellung
	\acro{AGGF}{Arbeitsgemeinschaft Grünland und Feldfutterbau}
	\acro{NEL}{Netto-Energie-Laktation}
\end{acronym}
\newpage

\section{Einleitung}
\label{sec:Einleitung}

Unter den strengeren Auflagen bzgl. der Düngung von landwirtschaftlich genutzten Flächen sowie einer Optimierung der Nutzung des Grünlandes in der Milchviehhaltung steigen die Anforderungen an das Grünland.
Insbesondere die \ac{NEL} Erträge stehen dabei im Fokus.
Die \ac{AGGF} hat sich im Rahmen Ihrer 63. Jahrestagung unter dem Motto "Grünland 2050" getroffen.
\textcite[33-36]{weggler2050leguminosen} haben sich mit der Möglichkeit der Steigerung des Leguminosen-Anteils und der Reduzierung der N-Düngung beschäftigt.

Da Leguminosen Stickstoff aus der Luft binden können sind diese nicht auf eine ausreichende N-Düngung angewiesen und sind gegenüber Gras bei intensiver N-Düngung nicht konkurrenzfähig.
Allerdings haben Leguminosen aufgrund ihrerer Stickstofffixierung sehr hohe Proteingehalte und können somit die \ac{NEL}-Erträge des Grünlandes steigern.
Damit die Grasnarbe gegenüber unerwünschten Pflanzen einen ausreichend konkurrenzfähig ist, ist eine ausreichende Stickstoffversorgung der Gräser sehr wichtig.
Somit ist ein Ausgleich der Interessen der Gräser und Leguminosen notwendig um die \ac{NEL}-Erträge zu optimieren.
Daher liegt hier ein klasisches Optimierungsproblem vor.

Die Einflüsse auf die Umwelt über eine geringere N-Düngung sind politisch gewollt.
Der Einfluss auf z.B. Nirtratauswaschungen wird nicht untersucht \parencite[33-36]{weggler2050leguminosen} und somit richtet sich der Artikel eindeutig an die Landwirtschaft und nicht an die Politik.

\section{Literatur}
\blindtext

\subsection{John B. Goodenough}
\blindtext

\subsubsection{Nobelpreis}
\Blindtext

\subsubsection{Li-Ion Batterie}
\Blindtext

\subsection{Alfred Nobel}
\blindtext

\section{Genial}
Das \ac{EMF} wird genutzt um Modelle einfach und automatisiert zu erstellen \parencite[348\psq]{karg1969einfluss}.
Nach \textcite[347]{helmert2003n} spielt \ac{EMF} hierbei eine wichtige Rolle. \ac{KIELER} wird an der \ac{CAU} entwickelt \parencite[34\psqq]{weggler2050leguminosen} wie \textcite[200]{karg1965larvalsystematische} beschrieben hat.
\ac{IUR} ist ein wichtiger bestandteil von \ac{SCCharts} \parencites[235]{karg1985zwei}[88]{bellebaum2008rohricht}.
\subsection{Wissenwertes}
\blindtext

\subsection{Details}
\blindtext

%Literaturverzeichnis
\begin{singlespace} %einfacher Zeilenabstand im Literaturverzeichnis
\begin{flushleft} %linksbündig
\addcontentsline{toc}{section}{Literaturverzeichnis}
\printbibliography[title=Literaturverzeichnis]
\end{flushleft}
\end{singlespace}

%Anhang
\end{document}
