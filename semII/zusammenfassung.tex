% !TEX root = twe2.tex

\section{Zusammenfassung}
\label{sec:Zusammenfassung}

Die Steigerung der \ac{NEL} Erträge von den Grünlandflächen ist für Milchvieh haltende Betriebe eine wichtige Aufgabe.
Eine Nachsaat mit Weiß- und Rotklee kann \ac{NEL} Erträge auch bei minimaler N Düngung etwas über dem Niveau einer Düngung mit 170 kg N ha\textsuperscript{-1}a\textsuperscript{-1} halten.
Dies ist unter dem Gesichtspunkten der höheren Auflagne an die N-Düngung eine wichtige Erkenntnis.
So können Milchviehbetriebe auf dem Grünland den Einsatz von N-Düngemittel potentiel deutlich reduzieren.
Allerdings erfolgt ein großteil der N-Düngung bereits über den Wirtschaftsdünger der Betriebe.
Eine Reduzierung der N-Düngung würde diese Betriebe dazu zwingen, einen Teil des Wirtschaftsdünger zu exportieren.
Bei dem momentanen Einsatz von Düngemittel scheint eine Steigerung von ca. 10\% möglich zu sein.
Das Potential von einer Steigerung von 25\% bei einer besseren Futterkonservierung ist nicht zu vernachlässigen.
