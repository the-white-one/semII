% !TEX root = twe2.tex

\section{Zusammenfassung}
\label{sec:Zusammenfassung}
Insbesondere die Rotkleenachsaat hat das Potential die \ac{NEL}-Erträge zu steigern.
Aufgrund der Steigerung der \ac{TM}-Erträge sinken bei einer Rotkleenachsaat die \ac{XP}- und \ac{NEL}-Gehalte des Aufwuchses.
Der Weißklee hat eine kleinere Steigerung der \ac{NEL}-Ertäge mit geringem Einfluss auf die \ac{NEL}-Konzentration erreicht.

Die zusätzliche N-Quelle stellt allerdings für viele Milchkuhbetriebe ein Problem da, da über das Kraftuffter in der Regel ein relativ hoher Eintrag an N vorhanden ist.
Somit könnte der betriebseigene Wirtschaftsdünger in der Regel nichtmehr nach den Vorschriften der \ac{DUV} innerhalb des Betriebes verwertet werden.
Aufgrund der gesenkten N-Düngung des Grünlandes würde sich das Problem sogar weiter verschärfen.

Falls sich die Rahmenbedingungen ändern sollte, sodass Kraftfutter als N-Quelle deutlich an Bedeutung verlieren würde, wäre die Nachsaat von Leguminosen eine N-Quelle welche erschlossen werden könnte.
Unabhängig von dem Aufwuchs sollte die Effizienz der Konservierung verbessert werden.
Verluste von ca. 20\% von der Mahd bis zum Futtertisch ist ein nicht unerheblicher Kostenfaktor welcher sicherlich optimiert werden kann.

%Die Steigerung der \ac{NEL} Erträge von den Grünlandflächen ist für Milchvieh haltende Betriebe eine wichtige Aufgabe.
%Eine Nachsaat mit Weiß- und Rotklee kann \ac{NEL} Erträge auch bei minimaler N Düngung etwas über dem Niveau einer Düngung mit 170\,kg\,N\,ha\textsuperscript{-1}a\textsuperscript{-1} halten.
%Dies ist unter dem Gesichtspunkten der höheren Auflagne an die N-Düngung eine wichtige Erkenntnis.
%So können Milchviehbetriebe auf dem Grünland den Einsatz von N-Düngemittel potentiel deutlich reduzieren.
%Allerdings erfolgt ein großteil der N-Düngung bereits über den Wirtschaftsdünger der Betriebe.
%Eine Reduzierung der N-Düngung würde diese Betriebe dazu zwingen, einen Teil des Wirtschaftsdünger zu exportieren.
%Bei dem momentanen Einsatz von Düngemittel scheint eine Steigerung von ca. 10\% möglich zu sein.
%Das Potential von einer Steigerung von 25\% bei einer besseren Futterkonservierung ist nicht zu vernachlässigen.
