% !TEX root = twe2.tex

\section{Literatur}
\label{sec:Literatur}

Aufgrund der preiswerten Versorgung von Proteinen über importiertes Soja ist die Steigerung der \ac{NEL}-Erträge zu Lasten der \ac{TM}-Erträge bisher relativ uninteressant gewesen.
Inzwischen verlangen aber immer mehr Verbraucher Produkte welche ohne den Einsatz von Gentechnik veränderten Pflanzen hergestellt werden.
Dadurch ist der Einsatz von importierten Soja für einige Betriebe nicht mehr möglich und diese Milcherzeuger müssen daher andere Proteinquellen erschließen.

\subsection{Einfluss auf TM-Erträge}
\label{subsec:TM}

Die Grasbestände sind insbesondere auf einen hohen \ac{TM}-Ertag ausgelegt.
Dieser wird über sehr ertragsreiche, aber auch auf N-Düngung angewiesenen Arten und Sorten erreicht.
Leguminosen reduzierne die N\textsubscript{2}-Fixierung bei N-Düngung \parencite{ledgard2001nitrogen}, weswegen die N-Düngung zumindestens reduziert werden muss \parencite[34]{weggler2050leguminosen}.
Bei einer Reduzierung der N-Düngung um die Leguminosen in den Bestand zu integrieren ist daher zu befürchten, dass die \ac{TM}-Erträge absinken werden.
Dies ist problematisch, da die Milcherzeuger dann mehr Fläche brauchen um ihre Tiere zu versorgen.
Eine Verkleinerung des Tierbestandes ist aufgrund der hohen Abschreibungen in der Innenwirtschaft häufig nicht kurzfristig wirtschaftlich darstellbar.

\subsection{Einfluss auf Proteingehalt}
\label{subsec:Protein}

Auch wenn Leguminosen generell die Möglichkeit haben höhere \ac{XP}gehalte zu generieren, stellt sich die Frage ob eine Nachsaat von Leguminosen ausreicht um die \ac{XP}gehalte bei einer Reduizierung der N-Düngung konstant zu halten.
Nach \textcite[35]{weggler2050leguminosen} ist es möglich die \ac{XP}gehalte zu steigern.

\subsection{Mineralische N-Düngung}
\label{subsec:Lit:N-Düngung}

Da in der Milchviehhaltung generell eine ausreichende Menge Gülle anfällt, ist davon auszugehen, dass die Grünlandflächen den größten Teil ihrer Düngung über die Gülle bekommen.
Insbesondere die Stickstoffversorgung ist derzeit eher unproblematisch aufgrund der über das Kraftfutter in den Kreislauf eingebrachten Eiweiße.
Daher wird derzeit nur ein kleiner Teil der Stickstoffversorgung des Grünlandes über mineralischen Dünger abgebildet.
Dies ist problematisch, da die organische Düngung nur mit relativ hohem Aufwand reduziert werden kann.
Die Gülle muss an andere Betriebe abgegeben werden und mineralischer Phosphor- und insbesondere Kalidünger muss den Bedarf der Pflanzen decken.
Bei einer Reduzierung der Eiweißkonzentration im Kraftfutter ist davon auszugehen, dass der Stickstoffgehalt der Gülle auch absinkt.

\subsection{Ernteverluste}
\label{subsec:Lit:Ernte}

Nachdem die \ac{XP} erfolgreich auf dem Feld produziert wurden, müssen diese konserviert werden.
Jeder Verlust von \ac{XP} in der Ernte muss entweder über den Zukauf oder über geringere Milchleistung bezahlt werden.
Über lagen Feldliegezeiten, hohe Bröckelverluste und ähnliches steigen die \ac{XP}-Verluste.

\subsubsection{Silage}
\label{subsub:Silage}
Bei der Silierung treten Silierungsverluste auf, desweiteren können bei nicht ausreichender Verdichtung, kein Luftabschluss oä Fehlgärungen auftreten.
Die Ernteverluste betragen etwa 22\% \parencite[30]{fritz2018wirtschaftliche}.


\subsubsection{Grascobs}
\label{subsub:Peletts}
Sehr aufwendig, geringe Verluste \parencite[12f]{engel2013protein}

\subsubsection{Heu}
\label{subsub:Heu}
Lange Feldliegezeit oder teure Heutrocknung, Ernteverluste \ac{NEL} zwischen 33\% und 21\% \parencite[30]{fritz2018wirtschaftliche}.

