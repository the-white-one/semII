% !TEX root = twe2.tex

\section{Literatur}
\label{sec:Literatur}

Aufgrund der preiswerten Versorgung von Proteinen über importiertes Soja ist die Steigerung der \ac{NEL}-Erträge zu Lasten der \ac{TM}-Erträge bisher relativ uninteressant gewesen.
Inzwischen verlangen aber immer mehr Verbraucher Produkte welche unter Gesichtspunkten des Umweltschutzes und Ressourcenschonung produziert wurden.
Dadurch ist der Einsatz von importierten Soja für einige Betriebe nicht mehr möglich bzw. erschwert die Vermarktung der Milch und diese Milcherzeuger müssen daher andere Proteinquellen erschließen.

\subsection{Einfluss auf TM-Erträge}
\label{subsec:TM}

Die Grasbestände sind insbesondere auf einen hohen \ac{TM}-Ertag ausgelegt.
Dieser wird über sehr ertragsreiche, aber auch auf N-Düngung angewiesenen Arten und Sorten erreicht.
Allerdings hat sich bereits gezeigt, dass eine Artenreiche Gräsermischung, insbesondere mit Leguminosen, höhere Erträge liefern können als Reinsaaten der ertragreichsten Art \parencite{nyfeler2009strong}.\todo{Seitenzahl einfügen!}
Leguminosen reduzierne die N\textsubscript{2}-Fixierung bei N-Düngung \parencite{ledgard2001nitrogen}\todo{Seitenzahl einfügen}, weswegen die N-Düngung zumindestens reduziert werden muss \parencite[34]{weggler2050leguminosen}.

Bei einer Reduzierung der N-Düngung um die Leguminosen in den Bestand zu integrieren ist daher zu befürchten, dass die \ac{TM}-Erträge absinken werden.
Dies ist zu vermeiden, da viele Betriebe darauf angewiesen sind, mit den vorhanden Flächen ausreichend Futter für Ihre Tiere zu produzieren.
Nach \textcite[11]{engel2013protein} sind die Verluste an \ac{TM} Ertrag bei einer Reduzierung der N-Düngung von 240 kg N ha\textsuperscript{-1}a\textsuperscript{-1} ohne Leguminosen auf 80 kg N ha\textsuperscript{-1}a\textsuperscript{-1} mit Weißklee bei etwa 20\%.

Allerdings ist auch zu erwähnen, dass die \ac{EU} eine Düngung von 240 kg N ha\textsuperscript{-1}a\textsuperscript{-1} eventuell im Zuge der gemeinsamen Agrarpolitik verboten oder eingeschränkt könnte.
Daher scheint der Vergleich von 160 kg N ha\textsuperscript{-1}a\textsuperscript{-1} ohne Leguminosen mit 80 kg N ha\textsuperscript{-1}a\textsuperscript{-1} mit einem Leguminosenanteil unter langfristigen Gesichtspunkten angebrachter.
Da betriebseigener Wirtschaftsdünger i.d.R. kostengünstig zur Verfügung steht, ist eine gewisse N-Düngung wirtschaftlich nur schwer zu vermeiden.
In diesem Fall reduziert sich der \ac{TM} Ertrag nach \textcite[11]{engel2013protein} um etwa 2,5\%.
Leider sind keine Konfidenzintervalle angegeben, sodass eine Aussage über die statistische Signifikanz hier leider nicht möglich ist.
Allerdings ist eine Abweichung von 2,5\% auch relativ gering und wird in der Praxis wohl auch kaum bemerkt werden.

Nach \textcite[35-36]{weggler2050leguminosen} sind die \ac{NEL} Erträge je Hektar bei Rotklee größer als bei Weißklee, allerdings sinkt bei der Rotkleevariante der \ac{NEL}-Gehalt des Aufwuchses.
Dies bedeutet, dass mit einer Rotkleenachsaat nicht nur der \ac{NEL} Ertrag gesteigert werden kann, sondern auch der \ac{TM} Ertrag.
Allerdings geht dies zu Lasten des \ac{NEL}- und \ac{XP}-Gehalts welches den Futterwert deutlich senkt.
Da die Futteraufnahme der Tiere begrenzt ist, ist mit einem Rückgang der Milchleistung aus dem Grundfutter zu rechnen.

\subsection{Einfluss auf Rohproteingehalt}
\label{subsec:Protein}

Auch wenn Leguminosen generell die Möglichkeit haben höhere \ac{XP}-Gehalte zu generieren, stellt sich die Frage ob eine Nachsaat von Leguminosen ausreicht um die \ac{XP}-Gehalte bei einer Reduizierung der N-Düngung konstant zu halten.
Nach \textcite[11]{engel2013protein} ist es bei reduzierter N-Düngung möglich die \ac{XP}-Gehalte leicht zu steigern.
Natürlich hängen die \ac{XP}-Erträge und -Gehalte von weiteren Faktoren ab, welche eventuell auch in eine Wechselwirkung treten können, hier aber nicht berücksichtigt werden können.

Generell setzen Pflanzen N in \ac{XP} um, um über die Chloroplasten Photosynthese betreiben zu können.
Daher ist eine ausreichende N-Versorgung essentiell für hohe Erträge.\todo{Hier ist noch ein bisschen was zu schreiben, evtl noch ein bisschen was zu lesen}


\subsection{Stickstoff in der Milchproduktion}
\label{subsec:Stickstoff}
Betriebe sind von der \ac{EU} verpflichtet über, z.B. die Hoftorbilanz die N-Bilanz des Betriebes zu berechnen.
Viele Betriebe haben dabei positive Saldi \parencite[7ff]{lellmann2005untersuchungen} und müssen daher weitere N-Quellen meiden.
Eine große, und in den meisten Fällen außerbetrieblieche Quelle, ist das Kraftfutter \parencite[62]{lellmann2005untersuchungen}.

Stickstoffverluste treten zum einen über den Verkauf von Milch über das Milcheiweiß auf.
Zum anderen entstehen beim Wirtschaftsdünger unvermeidbare N-Verluste bei der Lagerung und Ausbringung.
Deweiteren entstehen auf den Feldern, insbesondere bei zu hoher N-Düngung, Nitratauswaschungen.
Diese stehen immer stärker in der Kritik, da diese das Grundwasser erreichen können und Nitrat im Körper zum Nitrit umgewandelt wird.
Da Nitrit giftig ist, hat die \ac{EU} einen strengen Grenzwert von 50mgl\textsuperscript{-1} Nitrat im Grundwasser festgelegt.\todo{Quelle Nitratauswaschungen}


\subsection{Ernteverluste}
\label{subsec:Lit:Ernte}

Nachdem die \ac{XP} erfolgreich auf dem Feld produziert wurden, müssen diese konserviert werden.
Jeder Verlust von \ac{XP} in der Ernte muss entweder über den Zukauf oder über geringere Milchleistung bezahlt werden.
Über lagen Feldliegezeiten, hohe Bröckelverluste und ähnliches steigen die \ac{XP}-Verluste.

\subsubsection{Silage}
\label{subsub:Silage}
Bei der Silierung treten Silierungsverluste auf, desweiteren können bei nicht ausreichender Verdichtung, kein Luftabschluss oä Fehlgärungen auftreten.
Die Ernteverluste betragen etwa 22\% \parencite[30]{fritz2018wirtschaftliche}.


\subsubsection{Grascobs}
\label{subsub:Peletts}
Sehr aufwendig, geringe Verluste \parencite[12f]{engel2013protein}

\subsubsection{Heu}
\label{subsub:Heu}
Lange Feldliegezeit oder teure Heutrocknung, Ernteverluste \ac{NEL} zwischen 33\% und 21\% \parencite[30]{fritz2018wirtschaftliche}.

