% !TEX root = twe2.tex

\section{Diskussion}
\label{sec:Disukussion}
Steigerung der \ac{NEL} Erträge.
Im Vergleich zu den Ernteverlusten (von der Mahd bis zum Futtertisch) von ca. 20\% \cref{subsub:Silage} ist ein sehr großes Potential darin vorhanden, diese Verluste zu minimieren.
\textcite[33-36]{weggler2050leguminosen} hat Möglichkeiten zur Steigerung der \ac{NEL} Erträge im Grünland aufgezeigt, allerdings ist zu beachten, dass aktuelle Verfahren der Futterkonxerierung zu hohen Verlusten führen.
Mit zukünftigen Ernte- und Konservierungsverfahren werden Landwirte hoffentlich Möglichkeiten haben die Verluste zu reduzieren.

\subsection{Literaturkritik}
\label{sub:kritik}
In \textcite[35]{weggler2050leguminosen} ist die Legende sowie Achsbeschriftung der Abb. 1 nicht korrekt umgesetzt.
So wird zum Beispiel der \ac{NEL} Gehalt mit GJ ha\textsuperscript{-1} beschrfitet, statt MJ kg\textsuperscript{-1} in \ac{TM}.

\subsection{Leguminosen als Proteinlieferant}
\label{sub:leguminosen}
Wie in \ref{subsec:Protein} gezeigt, können Leguminosen den \ac{XP} Ertrag vom Grünland erhöhen.
Der Versuch von \textcite[33-36]{weggler2050leguminosen} wurde allerdings nur bis zu einer Düngung von 170kg N ha\textsuperscript{-1} a\textsuperscript{-1} gesteigert.
Daher ist es schwierig, einen Vergleich zwischen einer, in Norddeutschland üblichen, N-Düngung von über 200kg N ha\textsuperscript{-1} sowie einer minimierten N-Düngung mit Leguminosen zu ziehen.
Eine Nachsaat mit Rotklee hatte häufig einen leicht negativen Einfluss auf den \ac{XP} Gehalt des Aufwuchs, während eine Weißlkleenachsaat tendenziell eine leichte Steigerung der \ac{XP} Gehalte zur Folge hatte.
Generell sind deutliche Steigerungen des \ac{NEL} Ertrages möglich, insbesondere bei (stark) eingeschränkter N-Düngung.

\subsection{Effizienz der Konservierung}
\label{sub:konservierung}
\ac{NEL} Verluste während der Ernte, Konservierung oder Lagerung sind besonders kritisch zu betrachten.
Nachdem der Landwirt aufwändig hochwertiges Futter erzeugt hat, verliert dieses etwa 20\% des \ac{NEL} Ertrags.
Diese Verluste sind aus betriebswirtschaftlicher Sicht langfristig nicht zu rechtfertigen.
Die Reduzierung dieser Verluste wird immer wichtiger, da vermutlich größere Anteile der \ac{NEL} über das Grundfutter abgedeckt werden muss.
Eine Verbesserung der Ernteverfahren bzw. Konservierungsmethoden ist daher dringend geboten.
Es bleibt somit zu hoffen dass die Forschung neue wirtschaftliche Verfahren entwickelt welche eine wirtschaftliche Nutzung des kompletten \ac{NEL} Ertrags von landwirtschaftlichen Flächen erlaubt.

\subsection{Fazit}
\label{subsec:fazit}
Leguminosen sind eine sinnvolle Variante um die \ac{NEL} Erträge des Grünlandes zu steigern.
Neben der Steigerung der Erträge wäre eine effizientere Verwertung dieser wünschenswert.
