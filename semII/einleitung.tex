% !TEX root = twe2.tex


\section{Einleitung}
\label{sec:Einleitung}

Unter den strengeren Auflagen bzgl. der Düngung von landwirtschaftlich genutzten Flächen sowie einer Optimierung der Nutzung des Grünlandes in der Milchviehhaltung steigen die Anforderungen an das Grundfutter.
Ein entscheidenes Qualitätsmerkmal der Grassilage in der Milchkuhfütterung ist der \ac{NEL}-Gehalt.
Ein höherer \ac{NEL}-Gehalt im Grundfutter geht mit einer höheren Milchleistung einher.
Somit ist eine Steigerung der \ac{NEL}-Erträge der Grundfutterproduktion auch eine potentielle Steigerung des Milchertrags.
Die \ac{AGGF} hat sich 2019 im Rahmen Ihrer 63. Jahrestagung unter dem Motto "Grünland 2050" getroffen um über zukünftige Entwicklungen des Grünlandes zu diskutieren.
\textcite[33-36]{weggler2050leguminosen} haben sich mit der Möglichkeit der Steigerung des Leguminosen-Anteils und der Reduzierung der N-Düngung beschäftigt.

Da Leguminosen molekularen Stickstoff (N\textsubscript{2}) aus der Luft binden können, sind diese, im Gegensatz zu den Gräsern, nicht auf eine ausreichende N\textsubscript{min}-Versorgung angewiesen.
Somit können Leguminosen auch ohne oder relativ geringer N-Düngung relativ hohe \ac{TM}-Eträge bei hohen \ac{XP}-Gehalten erzielen.
Dies wirft die Frage auf, ob ein bestimmter Leguminosenanteil in der Gräsermischung bei gleichzeitiger Reduktion der N-Düngung in der Lage ist höhere \ac{NEL}-Erträge zu liefern.

Die gesteigerten \ac{NEL}-Erträge sollen natürlich auch auf dem Futtertisch ankommen.
In Norddeutschland ist die Grassilage die übliche Konservierungsmthode von Grünlandbeständen.
Allerdings treten dabei etwas über 20\% \ac{NEL}-Verluste auf \parencite[30]{fritz2018wirtschaftliche}.

Zielsetzung.\todo{Zielsetzung vernünftig ausschreiben}
