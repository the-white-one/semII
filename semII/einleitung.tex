% !TEX root = twe2.tex


\section{Einleitung}
\label{sec:Einleitung}

Unter den strengeren Auflagen bzgl. der Düngung von landwirtschaftlich genutzten Flächen sowie einer Optimierung der Nutzung des Grünlandes in der Milchviehhaltung steigen die Anforderungen an das Grünland.
Insbesondere die \ac{NEL}-Erträge \todo{wie Abkürzungen in zusammengesetzten Wörtern einführen?} stehen dabei im Fokus.
Die \ac{AGGF} hat sich im Rahmen Ihrer 63. Jahrestagung unter dem Motto "Grünland 2050" getroffen.
\textcite[33-36]{weggler2050leguminosen} haben sich mit der Möglichkeit der Steigerung des Leguminosen-Anteils und der Reduzierung der N-Düngung beschäftigt.

Da Leguminosen Stickstoff aus der Luft binden können sind diese nicht auf eine ausreichende N-Düngung angewiesen und sind gegenüber Gras bei intensiver N-Düngung nicht konkurrenzfähig.
Allerdings haben Leguminosen aufgrund ihrerer Stickstofffixierung sehr hohe Proteingehalte ohne dabei auf eine intensive N-Düngung angewiesen zu sein.
Dies wirft die Frage auf, ob ein bestimmter Leguminosenanteil in der Gräsermischung bei gleichzeitiger Reduktion der N-Düngung in der Lage ist höhere \ac{NEL}-Erträge zu liefern.
Damit die Grasnarbe gegenüber unerwünschten Pflanzen einen ausreichend konkurrenzfähig ist, ist eine ausreichende Stickstoffversorgung der Gräser sehr wichtig.
Somit ist ein Ausgleich der Interessen der Gräser und Leguminosen notwendig um die \ac{NEL}-Erträge zu optimieren.
Daher liegt hier ein klasisches mehrdimensionales Optimierungsproblem vor.

Die Einflüsse auf die Umwelt über eine geringere N-Düngung sind politisch gewollt.
Dies wird nicht untersucht und somit richtet sich der Artikel eindeutig an die Landwirtschaft und nicht an die Politik.
