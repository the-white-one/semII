% !TEX root = twe2.tex


\section{Einleitung}
\label{sec:Einleitung}

Unter den strengen Auflagen der \ac{DUV} von landwirtschaftlich genutzten Flächen sowie gesellschaftlicher Kritik an der Fütterung von Soja aus Südamerika in der Milchkuhhaltung steigt die Bedeutung des Grundfutters.
Ein entscheidenes Qualitätsmerkmal des Grundfutters in der Milchkuhfütterung ist der \ac{NEL}-Gehalt.
%Ein höherer \ac{NEL}-Gehalt im Grundfutter geht mit einer höheren Milchleistung einher.
Somit ist eine Steigerung der \ac{NEL}-Erträge des Grünlandes ein wichtiges Ziel der meisten Milchkuhhalter.
Die \ac{AGGF} hat sich 2019 im Rahmen Ihrer 63. Jahrestagung unter dem Motto "Grünland 2050" \space getroffen um über zukünftige Entwicklungen sowie Nutzuungen des Grünlandes zu diskutieren.
\textcite[33-36]{weggler2050leguminosen} haben sich mit der Möglichkeit befasst den Leguminosen-Anteils zu steigern, um über einen gesteigerten \ac{XP}-Ertrag den \ac{NEL}-Ertrag zu steigern.

Da Leguminosen molekularen Stickstoff (N\textsubscript{2}) aus der Luft binden können, sind diese, im Gegensatz zu den Gräsern, nicht auf eine N\textsubscript{min}-Versorgung angewiesen.
Somit können Leguminosen auch ohne oder relativ geringer N-Düngung relativ hohe \ac{TM}-Eträge produzieren.
Dies wirft die Frage auf, ob ein bestimmter Leguminosenanteil in der Gräsermischung bei gleichzeitiger Reduktion der N-Düngung in der Lage ist zumindestens vergleichbare oder höhere \ac{NEL}-Erträge zu liefern.

Die \ac{NEL}-Erträge sollen natürlich auch auf dem Futtertisch ankommen.
In Norddeutschland ist die Grassilage die übliche Konservierungsmthode von Grünlandbeständen, allerdings treten dabei etwa 20\% \ac{NEL}-Verluste auf \parencite[30]{fritz2018wirtschaftliche}.

Somit exisitieren viele Möglichkeiten mit welchen der Landwirt auf die Leistungsfähigkeit seines Grünlandes Einfluss nehmen kann.
Der Fokus liegt dabei auf den \ac{NEL}-Erträgen und -Gehalten, die \ac{TM}-Erträge und \ac{XP}-Erträge und -Gehalte werden auch betrachtet.


%Zielsetzung.\todo{Zielsetzung vernünftig ausschreiben}
